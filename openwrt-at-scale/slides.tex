\documentclass{beamer}

\usetheme[progressbar=frametitle]{metropolis}
%\setbeamertemplate{frame numbering}[fraction]
\useoutertheme{metropolis}
\useinnertheme{metropolis}
\usefonttheme{metropolis}
\usecolortheme{spruce}
\setbeamercolor{background canvas}{bg=white}

\usepackage{listings}
% storing our images
\usepackage{graphicx}
\graphicspath{{./images/}}

%Information to be included in the title page:
\title{Building SCaLE's OpenWrt Conference WiFi}
\subtitle{Yes...The very network you're connected to right now}
\author{
  Robert Hernandez \\
  CTO @ Nebulaworks \\
  @sarcasticadmin
}
\date{July 2022}

\setbeameroption{hide notes} % Only slides
%\setbeameroption{show only notes} % Only notes
%\setbeameroption{show notes on second screen=right} % Both

%\setbeamertemplate{note page}{\pagecolor{yellow!5}\insertnote}\usepackage{palatino}

\begin{document}
\metroset{block=fill}
\frame{\titlepage}

\begin{frame}
\frametitle{Tech Team}
  %{\large \textbf{SCaLE Tech Team}}
  \begin{center}
    \includegraphics[scale=.5]{scale-tech}
  \end{center}
  \note[item]{SCaLE is run by many volunteers and traditionally the scale-network team provides this service}
  \note[item]{Our tech team is made up by 10 dedicated team members who have been dedicated to the conference for many years}
\end{frame}

\begin{frame}
\frametitle{Why}
  \begin{center}
  {\large \textbf{Conference attendees need WiFi} \\[20pt]}
  \includegraphics[scale=.5]{wifi-signage}
  \end{center}
  \note[item]{And countless options to choose from: many propertary and only a few open}
  \note[item]{SCaLE is run by many volunteers and traditionally the scale-network team provides this service}
\end{frame}


\begin{frame}
\frametitle{History}
  % Alert to highlight each part of the timeline
  \begin{itemize}[<alert@+>]
    % TODO: What about prior years?
    \item Jan 2017 - "Does anyone know how to build OpenWrt?"
  \note[item]{During one of my first conference calls on the tech team prior to SCaLE 15x in Jan 2017}
  \note[item]{Previous Tech Team was no more and everyone was new for the coming year}
  \note[item]{Bare minimum for SCaLE 15x but config was kept}
    \onslide<2->{
    \item March 2017 - SCaLE 15x
    }
  \onslide<3->{
  \item Sept 2017 - Decision to stick with OpenWrt
  }
  \onslide<4->{
  \item July 2022 - SCaLE 19x
  }
  \end{itemize}
  \note[item]{This led to 5 years of iteration by the entire Tech Team}
  \note[item]{Many of the things Ill be covering were built out of necessity and in an effort to provide a reliable network come show time}
\end{frame}

% TODO: Where to put "how" might need to be before history
\begin{frame}
\frametitle{Hardware}
  \begin{center}
    {\large \textbf{ \textasciitilde 120 consumer "routers"} \\[20pt]}
    \small{
    \begin{tabular}{| l | l | l | l | l |}
      \hline
      Model & WiFi Chipsets & 2.4 GHz & 5 GHz & RAM \\ \hline
      Netgear 3700v2 & Ath AR9220 + AR9223  & b/g/n & a/n & 64MB \\ \hline
      Netgear 3800[CH] & Ath AR9220 + AR9223 & b/g/n & a/n & 64MB \\ \hline
    \end{tabular} \\[20pt]
    }
  \includegraphics[scale=.3]{wndr3700_v2}
  \end{center}
  \note[item]{Netgear 3700v2 and 3800 series}
  \note[item]{We do have support the TPlink C2600 but wireless driver support has been unreliable of the past couple of years}
\end{frame}

\begin{frame}
\frametitle{Repo}
  \begin{center}
  {\large \textbf{https://github.com/socallinuxexpo/scale-network}}
  \end{center}
  \note[item]{SCaLE techs single source of truth for all deployments}
\end{frame}


\begin{frame}
  \frametitle{Build}
  {\large \textbf{SCaLE's OpenWrt image}} \\[20pt]
  \begin{block}{16MB max}
    Kernel: Linux 5.10.89 2022 mips GNU/Linux \\
  Pkgs: Bash, Python, OpenSSH, lldp, tcpdump, rsyslog, apinger, zabbix \\
  Files: Openwrt Configs, ssh keys, etc.
  \end{block}
  Removed: luci, dropbear, dnsmasq, logd \\[20pt]

  \note[item]{openwrt is built of snapshots of upstream master for both kernel and packages}
  \note[item]{periodically bumped throughout the year}
  \note[item]{limited to 16MB on netgears}
  \note[item]{unnecessary packages and defaults are stripped out [luci, nat, etc.]}
  \note[item]{Bash, Python, OpenSSH, tcpdump, zabbix, rsyslog installed via opkg at image build time}
  \note[item]{mixin config files via templating process}
\end{frame}

% TODO: How to convey the image of massflash
\begin{frame}
  \frametitle{Flash}
    \only<1>{\line(1,0){50}}
    \only<2>{\textcolor{red}{Traditional}}
  \note[item]{manual setup, paperclip and detection of flash process}
    \only<3>{
    \begin{center}
      \includegraphics[scale=.25]{wifi-hard-reset}
    \end{center}
    }
    \only<4>{\textcolor{orange}{Solo}}
  \note[item]{expect script that waits for signals from AP that its in the proper mode}
    \only<5>{
      \textcolor{green}{Mass}
    }
    \only<9>{\textcolor{green}{Autoflash}}
  \note[item]{Tease this and mention that well discuss this last option later when we discuss testing}
  \only<1,2,4,5,9>{OpenWrt Flash}
  \only<6>{
    \begin{center}
      \includegraphics[scale=.25]{massflash-1}
    \end{center}
  }
  \only<7>{
    \begin{center}
      \includegraphics[scale=.25]{massflash-2}
    \end{center}
  }
  \only<8>{
    \begin{center}
      \includegraphics[scale=.25]{massflash-3}
    \end{center}
  }
  \note[item]{Requires openwrt and SSH access to the hardware}
  \note[item]{Via DHCP lease event then SCP and trigger reflash of hardware}
  \note[item]{Reflash fleet of 100 APs in 2 hours}
  \note[item]{Limited by switch ports and power strips}
  \note[item]{Consistent, prior we often werent sure if image was flashed}
\end{frame}

\begin{frame}
  \frametitle{Management}
  {\large \textbf{SCaLE's OpenWrt management} \\[20pt]}
  \begin{itemize}
    \item Custom DHCP options: WiFi Channels and Network Config
    \item Re-flash APs for any major system updates
    \item SSH
    \item Rsyslog and zabbix for observability 
  \end{itemize}
  \note[item]{Leverage custom DHCP options and scripts local to the AP}
  \note[item]{Network trunks and bridges are configured to accomediate all SCaLE needs (N buildins, etc.)}
  \note[item]{WiFi Channel info, network config provided via DHCP options}
  \note[item]{Anything adhoc is dealt with via SSH}
  \note[item]{Reflash APs for any major changes to configuration}
  \note[item]{mixin config files}
\end{frame}

\begin{frame}[label=foo]
  \frametitle{Automated Testing}
    \only<1>{
      {\large \textbf{Golden Tests}} \\[20pt]
      \lstinputlisting[language=bash]{snippets/golden.sh}
    }
    \only<2-4>{
      {\large \textbf{Weekly Image Build}} \\[20pt]
    }
    \only<3>{
      \includegraphics[scale=.1]{build-weekly-2}
    }
    \only<4>{
      \includegraphics[scale=.1]{build-weekly-1}
    }
    \note[item]{Ensure that existing upstream sources still build correctly}
    
  %\item<1,2> 
  %\note[item]{Check templated configuration against expected outputs in our CI}
    \only<5>{
      {\large \textbf{Serverspec}} \\[20pt]
      \lstinputlisting[language=ruby]{snippets/serverspec.rb}
    }
    \only<6>{
      {\large \textbf{Autoflash}} \\[20pt]
    }
    \only<7>{
      {\large \textbf{Autoflash}} \\[20pt]
      \begin{block}{Configuration}
      \small
      Hardware: Raspberry Pi 4, Relay, USB Ethernet Adapter, AP Board  \\
      Software: FreeBSD 13, GitLab runner, magic wormhole, and expect
      \end{block}
    }
    \only<8>{
      \includegraphics[scale=.05]{autoflash-1}
    }
    \only<9>{
      \includegraphics[scale=.05]{autoflash-2}
    }
    \only<10>{
      \includegraphics[scale=.05]{autoflash-3}
    }
\end{frame}

\begin{frame}
  \frametitle{Radio Testing}
  \begin{enumerate}
      \item iperf
      \item Dogfooding
      \item Surveying
  \note[item]{WiFi driver performance and stability is tested by contributors dogfooding the configuration throghout the year}
  \end{enumerate}
\end{frame}

\begin{frame}
  \frametitle{Impact}
  \begin{itemize}
    \item Stable platform for the conference
    \item Saves an entire day during show prep 
    \item Autoflash decreased feedback loops from 20+min to 5 min per test
    \item 52 serverspec tests being checked on all images
    \item Enables expanding AP hardware to support more than 1 model with limited overhead
  \end{itemize}
  \note[item]{}
\end{frame}

\begin{frame}
  \frametitle{Errata @ 19x}
  \begin{itemize}
    \item WiFi radios remaining up if initial boot had no connectivity
    \item DHCP reservations mismatches and range overlap
    \item DNS ACL assumptions for Pasadena Convention Center
  \end{itemize}
  \note[item]{}
\end{frame}

\begin{frame}
  \frametitle{Future}
  \begin{itemize}
    \item Belkin RT3200: WiFi 6
    \item DAWN: https://github.com/berlin-open-wireless-lab/DAWN
    \item Automated client WiFi testing in CI
    \item Advanced surveying process
  \end{itemize}
  \note[item]{DAWN works best with networks where AP and devices have 802.11k capabilities for discovering the quality of radio signals, plus 802.11v for requesting devices to move to a different AP.}
\end{frame}

\begin{frame}
\frametitle{Repo}
  \begin{center}
  {\large \textbf{https://github.com/socallinuxexpo/scale-network}}
  \end{center}
  \note[item]{SCaLE techs single source of truth for all deployments}
\end{frame}

\begin{frame}
\frametitle{This talk}
  \begin{center}
  {\large \textbf{https://github.com/sarcasticadmin/talks}}
  \end{center}
\end{frame}

\begin{frame}
\frametitle{End}
  \begin{center}
  {\large \textbf{Thank you}}
  \end{center}
\end{frame}

\end{document}
