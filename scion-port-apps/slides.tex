\documentclass[aspectratio=169]{beamer}
\usepackage[utf8]{inputenc}
\usepackage[T1]{fontenc}
\usetheme[darkmode, nofooter]{pureminimalistic}
%\setbeamertemplate{frame numbering}[fraction]
%\useoutertheme{pureminimalistic}
%\useinnertheme{pureminimalistic}
%\usefonttheme{pureminimalistic}
%\usecolortheme{spruce}
%\setbeamercolor{background canvas}{bg=white}

\usepackage{listings}
% storing our images
\usepackage{graphicx}
\graphicspath{{./images/}}

%Information to be included in the title page:
\title[Default Options]{The NATS Message Bus on the SCiON Internet Architecture }
\author{
  Robert Hernandez \\
  @sarcasticadmin
}
\date{March 2023}

\setbeameroption{hide notes} % Only slides
%\setbeameroption{show only notes} % Only notes
%\setbeameroption{show notes on second screen=right} % Both

%\setbeamertemplate{note page}{\pagecolor{yellow!5}\insertnote}\usepackage{palatino}

\renewcommand{\logotitle}{\includegraphics%
  [width=.2\linewidth]{images/martincoit.png}}
\renewcommand{\logoheader}{\includegraphics%
  [width=.5\linewidth]{images/martincoit.png}}
\renewcommand{\logofooter}{\includegraphics%
  [width=.15\linewidth]{images/martincoit.png}}

\begin{document}

\maketitle

\begin{frame}
  % WIP
  Imagine a world where you can control and verify the packets of your most critical distributed systems.
  \note[item]{}
\end{frame}

\begin{frame}
\frametitle{Agenda}
  \begin{itemize}[<alert@+>]
    \item Message Bus Overview
    \item SCiON Overview
    \item Porting NATS to SCiON
    \item Results \& Next Steps
  \end{itemize}
\end{frame}

\begin{frame}[c]
  \begin{center}
    \Huge{Message Bus}
  \end{center}
\end{frame}

\begin{frame}
%TODO: Add graphic of message bus
\frametitle{What is a Message Bus?}
  Infrastructure that allows different systems to communicate through a shared interfaces via messages
  \note[item]{Uses a publish/subscribe model, subscribers and senders dont need to be aware of each other, FIFO ordering isnt guaranteed}
\end{frame}

\begin{frame}
%TODO: Add graphic of message bus
\frametitle{Message Bus Benefits}
  % Refs: https://www.youtube.com/watch?v=ZwZvQIuX0AU
  \begin{itemize}[<alert@+>]
    \item A common communication platform for messages
    \note[item]{}
    \onslide<2->{
    \item Authentication \& Authorization
    }
  \onslide<3->{
  \item Congestion Control
  }
  \onslide<4->{
  \item Prioritization
  }
  \end{itemize}
\end{frame}

\begin{frame}[c]
  \begin{center}
    \Huge{Challenges when operating a message bus}
  \end{center}
\end{frame}

\begin{frame}
  \frametitle{Network Security}
    \begin{columns}[t]
      \begin{column}{0.48\linewidth}
        \pause
        Challenge:
        \begin{itemize}
            \item<3-> Cluster Communications
            \item<5-> Client Authenticity
        \end{itemize}
    \end{column}
    \begin{column}{0.48\linewidth}
        Solutions:
        \pause
        \begin{itemize}
            \item<4-> Private Networks, VPNs, Whitelists
            \item<6-> PKI, Shared Tokens
        \end{itemize}
    \end{column}
  \end{columns}
  \note[item]{Uses a publish/subscribe model, subscribers and senders dont need to be aware of each other, FIFO ordering isnt guaranteed}
\end{frame}

\begin{frame}
\frametitle{CAP Theorem}
  No distributed system is safe from network failures, thus network partitioning generally has to be tolerated. In the presence of a partition, one is then left with two options: consistency or availability.
  \newline
  \begin{itemize}[<alert@+>]
    \item Consistency
    \note[item]{}
    \onslide<2->{
    \item Availability
    }
  \onslide<3->{
  \item Partition Tolerance
  }
  \end{itemize}
\end{frame}

\begin{frame}[c]
  \begin{center}
    \Huge{What if we could make the network more reliable and secure?}
  \end{center}
\end{frame}

\begin{frame}[c]
  \begin{center}
    %add logo
    \Huge{SCiON}
  \end{center}
\end{frame}

\begin{frame}
  \frametitle{What's SCiON?}
  SCION is the first clean-slate Internet architecture designed to provide route control, failure isolation, and explicit trust information for end-to-end communication.
\end{frame}

\begin{frame}
\frametitle{SCiON Capabilities}
  \begin{itemize}[<alert@+>]
    \item Security: Authenticated Control Plane and Resilience Against Path Hijacks
    \note[item]{}
    \onslide<2->{
    \item Stability: Native multipath capability at the network level with rapid path failover
    }
  \onslide<3->{
  \item Control: Path-awareness for end posts
  }
  \onslide<4->{
  \item Protection: Hidden paths and sender-based path selection
  }
  \onslide<5->{
  \item Performance: SCiON applications can select best paths based on latency, bandwidth, and other criteria
  }
  \end{itemize}
\end{frame}

\end{document}
